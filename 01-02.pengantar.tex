PHP adalah bahasa pemrograman yang dijalankan di sisi \textit{server} dalam bentuk \textit{scripting}, artinya bahasa pemrograman ini tidak perlu di \textit{compile} terlebih dahulu untuk dapat dijalankan, kita cukup menyiapkan \textit{interpreter}-nya saja.

PHP biasanya digunakan untuk membangun sebuah aplikasi Web yang dinamis, dimana halaman dapat melakukan respon terhadap \textit{request} yang dilakukan oleh pengguna. 

PHP pun telah digunakan secara luas dan menjadi alternatif gratis dibandingkan menggunakan bahasa sejenis seperti ASP milik Microsoft.

Untuk memulai melakukan praktek bahasa pemrograman menggunakan PHP, maka kita perlu mempersiapkan perangkat pendukung. Cara yang mungkin paling mudah adalah kita menggunakan aplikasi paket yang di dalamnya sudah terdapat \textit{web server} yang mendukung PHP serta basis data yang akan digunakan, aplikasi yang mungkin dapat kita gunakan adalah :

\begin{enumerate}
	\item LAMP, yang sebetulnya adalah singkatan dari Linux, Apache, MySQL, dan PHP. Tentunya aplikasi ini ditujukan untuk sistem operasi Linux, yang menggunakan Apache sebagai \textit{web server} yang tentunya \textit{plugin} untuk mendukung bahasa PHP sudah ada di dalamnya, dan MySQL sebagai basis datanya.
	\item WAMP, adalah singkatan dari Windows, Apache, MySQL, dan PHP. Mirip seperti LAMP, hanya ini ditujukan bagi sistem operasi Windows.
	\item MAMP, adalah singkatan dari Mac, Apache, MySQL, dan PHP. Untuk aplikasi ini dikhususkan bagi sistem operasi Mac.
	\item XAMPP, yang ini mendukung ketiga sistem operasi di atas dengan kelebihan mampu untuk mengolah bahasa pemrograman Perl.
\end{enumerate}

Maka pilihan untuk praktek Web Programming 1 kita akan menggunakan XAMPP agar adaptasi antar sistem operasi lebih mudah. XAMPP dapat diunduh pada alamat \url{https://www.apachefriends.org}.

Aplikasi pendukung lain untuk melakukan kegiatan praktikum kita adalah sebagai berikut :

\begin{enumerate}
	\item Git. Aplikasi ini digunakan untuk melakukan \textit{versioning} sehingga kita lebih mudah dalam melakukan kontrol perubahan yang terjadi pada kode program yang kita bangun. Server yang kita gunakan untuk menyimpan repositori hasil \textit{versioning} kita ada di alamat \url{https://github.com}. Github ini gratis. Untuk aplikasi Git dapat kita unduh di alamat \url{https://git-scm.com/}
	\item Visual Studio Code. Aplikasi ini adalah \textit{editor} yang akan digunakan dalam kegiatan praktikum pada mata kuliah Web Programming 1. Aplikasi ini gratis dan dapat diunduh pada alamat \url{https://code.visualstudio.com/} dengan dukungan instalasi untuk 3 (tiga) sistem operasi yang banyak digunakan, yaitu Linux, Windows, dan MacOS.
	\item \url{www.000webhost.com}. Ini adalah layanan \textit{hosting} gratis yang mampu menjalankan \textit{script} PHP dengan fasilitas sistem basis data MySQL. Yang akan kita gunakan sebagai tempat aplikasi yang telah kita bangun sampai dengan akhir tatap muka mata kuliah ini.
\end{enumerate}